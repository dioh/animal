\documentclass{beamer}

\usetheme{Copenhagen}
%% \usenavigationsymbolstemplate{}
\setbeamertemplate{navigation symbols}{}

%% \usecolortheme[rgb={0.4,0.5,0.4}]{structure}
\usepackage{color}

% \usepackage[T1]{fontenc}
% \usepackage{libertine}
\usepackage[spanish]{babel}
\usepackage[utf8]{inputenc}
\usepackage{graphicx}
\usepackage{verbatim}
%\usepackage{wrapfig}


\title{TP - Seg Inf}
      
\author{Alfonso, Foguelman, Gutesman, Ispani}
\institute{DC - FCEyN - UBA}
\date[11.2013]{SegInf, 2c - 2013}



\begin{document}
	\begin{frame}
		\titlepage
	\end{frame}

        \section{Animal}

        \frame{
            \begin{block}{Quote}
                Animalazo es un framework de .. ppapapapa
            \end{block} 
        }
%% 
%%         \frame{ 
%%             \title{?`Qu\'e es?}
%%             \begin{itemize}
%%                 \item Un paradigma de desarrollo
%%                 \item Dise\~nado para el c\'omputo de datos de gran escala 
%%                 \item Donde la performance es un un aspecto importante
%%                 \item Con una arquitectura de bajo costo
%%                 \item Y un requerimiento fuerte: Escalabilidad lineal (doble procesamiento, mitad del tiempo).
%%             \end{itemize} 
%%         }
%% 
%%         \frame{
%%             Conceptos fuertes
%%             \begin{itemize}
%%                 \item Automatic parallelization and distribution
%%                 \item Fault-tolerance
%%                 \item I/O scheduling
%%                 \item Status and monitoring 
%%             \end{itemize} 
%%         }
%% 
%%         \subsection{Framework de programaci\'on}
%%         \frame{
%%             \begin{center}
%%                 \large
%%             !`Datos, Datos, Datos ! \\
%%             \Huge Google procesa 24 petabytes/d\'ia
%%         \end{center}
%%         }
%% 
%%         \begin{frame}[fragile]
%%             \large
%%             Map - Reduce
%% \begin{verbatim}
%%  > map (in_key, in_value) -> 
%%         list(out_key, intermediate_value) 
%% 
%%  >  reduce (out_key, list(intermediate_value)) -> 
%%         list(out_value) 
%% 
%% \end{verbatim}
%% 
%% Orientado a documentos, clave-valor.  
%%             
%% \end{frame}
%% 
%% \frame{
%%     \begin{center}
%%         \Large
%%         Un ejemplo \\
%% 		\includegraphics[scale=0.5]{imgs/map-reduce-datos.png} 
%%     \end{center}
%%     \url{http://docs.mongodb.org/manual/core/map-reduce}
%% }
%% 
%% \begin{frame}[fragile]
%%     \begin{center}
%%         \Large 
%%         Map
%%     \end{center}
%%     \begin{verbatim}
%%  map = function() {
%%      emit(this.cust_id, this.amount); 
%%  } 
%%     \end{verbatim}
%%     \begin{center}
%%         \includegraphics[scale=0.5]{imgs/map-reduce-datos-map.png} 
%%     \end{center}
%% \end{frame}
%% 
%% \begin{frame}[fragile]
%%     \begin{center}
%%         \Large 
%%         Reduce
%%     \end{center}
%%     \begin{verbatim}
%%  reduce = function(key, values) {
%%     return Array.sum(values);
%%  } 
%%     \end{verbatim}
%%     \begin{center}
%%         \includegraphics[scale=0.5]{imgs/map-reduce-datos-reduce.png} 
%%     \end{center}
%% \end{frame}
%% 
%% \begin{frame}[fragile]
%%     \small
%%     \begin{verbatim}
%%     > use sales
%%     > show collections
%%         orders
%%     > db.orders.mapReduce(map, reduce, order_totals);
%%     > show collections 
%%         orders
%%         order_totals
%%         
%%     \end{verbatim}
%%     \begin{center}
%% 		\includegraphics[scale=0.4]{imgs/map-reduce-datos.png} 
%%     \end{center}
%% 
%% \end{frame}
%% 
%% 
%%         \subsection{Arquitectura}
%%         \frame{
%%             Localizaci\'on de los datos para minimizar I/O sobre la red, brindando:
%%             \begin{itemize}
%%                 \item Replicaci\'on
%%                 \item Disponibilidad 
%%             \end{itemize} 
%%             \includegraphics[scale=0.3]{imgs/arch.png}
%%         }
%% 
%%         \begin{frame}\title{NFS}
%%             \begin{itemize}
%%                 \item El \emph{Network File System} es un protocolo que permite acceder a FS remotos como si fueran locales, utilizando RPC.
%%                 \item La idea es que un FS remoto se monta en algún punto del sistema local y las aplicaciones acceden a archivos de ahí, sin saber que son remotos.
%%                 \item Para poder soportar esto, los SO incorporan una capa llamada \emph{Virtual File System}.
%%                 \item Esta capa tiene \emph{vnodes} por cada archivo abiertos. Se corresponden con inodos, si el archivo es local. Si es remoto, se almacena otra información.
%%                 \item Así, los pedidos de E/S que llegan al VFS son despachados al FS real, o al \emph{cliente de NFS}, que maneja el protocolo de red necesario.
%%                 \item Si bien del lado del cliente es necesario un módulo de kernel, del lado del server alcanza con un programa común y corriente.
%%             \end{itemize}
%%         \end{frame}
%% 
%%         \begin{frame}\title{NFS (cont.)}
%%             \begin{center}
%%                 \includegraphics[height=0.4\textheight]{imgs/nfs.png}
%%             \end{center}
%%             \begin{itemize}
%%             \item Otros FS distribuidos funcionan de manera similar (en cuanto a su integración con el kernel).
%%             \item Notar que, desde cierto punto de vista, NFS no es 100\% distribuido, ya que todos los datos de un mismo ``directorio'' deben vivir físicamente en el mismo lugar.
%%             \item Hay FS 100\% distribuidos, como AFS o Coda.
%%             \item Han tenido un éxito relativo.
%%             \end{itemize}
%%         \end{frame}
%% 
%%         \frame{
%%             \title{Jobs}
%%             Framework de ejecuci\'on
%%             \begin{itemize}
%%                 \item Map
%%                 \item Combine
%%                 \item Reduce 
%%             \end{itemize} 
%%             !`Alguien debe orquestar la ejecuci\'on! \\
%%             \url{http://www.cs.berkeley.edu/~matei/talks/2009/msr_mapreduce_scheduling.pdf}
%%         }
%%         \frame{
%%             \title{Jobs (cont.)}
%%             \begin{center}
%%                 \includegraphics[height=0.4\textheight]{imgs/jobs.png}
%%             \end{center} 
%%             \begin{itemize}
%%                 \item Organizar los Jobs en subtareas
%%                 \item Minimizar tiempo idle de los nodos del cluster
%%                 \item Subtareas organizadas para aprovechar vecindad espacial (en el cluster)
%%                 \item Maximizar el uso de los nodos del cluster
%%             \end{itemize}
%%             Y si cada nodo puede ejecutar varias tareas, ?`c\'omo se asignan?
%%         }
%% 
%%         \section{Introduciendo Mongo}
%%         \subsection{?`Qu\'e es?}
%%         
%%         \frame{
%%             Una base de datos No-SQL que implementa:
%%             \begin{itemize}
%%                 \item MapReduce
%%                 \item Replicaci\'on y alta disponibilidad
%%                 \item Autosharding, escalabilidad horizontal
%%                 \item GridFS
%%             \end{itemize} 
%%         }
%% 
%% 
%%         \section{El TP}
%%         \subsection{?`Que aprenderemos?}
%%         \frame{
%%             Se espera que:
%%             \begin{itemize}
%%                 \item Aprendan a manejar el modelo de programaci\'on.
%%                     \begin{itemize}
%%                         \item !`Como algo uan pero un poquito m\'as!
%%                     \end{itemize}
%%                 \item Realicen un an\'alisis de arquitectura. 
%%                 \item Entiendan los criterios de an\'alisis para observar un sistema distribuido.
%%                 \item Todo en una VM. 
%%             \end{itemize}
%% 
%%         }
%% 
%%         \section{Referencias}
%%         \tiny
%% 
%%         \frame{
%%             \begin{itemize}
%%                 \item Data-Intensive Text Processing with MapReduce, Jimmy Lin and Chris Dyer University of Maryland, College Park
%% 
%%                 \item Mining  of  Massive Datasets Anand Rajaraman Jure Leskovec  Stanford Univ. Jeffrey D. Ullman
%%                 \item Data-Intensive Text Processing with MapReduce Jimmy Lin and Chris Dyer University of Maryland, College Park
%%                 \item The Little MongoDB Book Karl Seguin
%%                 \item \url{http://docs.mongodb.org/manual/}
%%                 \item \url{http://research.google.com/archive/mapreduce-osdi04-slides/index.html}
%%                 \item \url{http://static.googleusercontent.com/media/research.google.com/es//archive/mapreduce-osdi04.pdf}
%%                 \item \url{http://www.mongodb.com/dl/big-data}
%%                 \item \url{http://info.mongodb.com/rs/mongodb/images/MongoDB-Performance-Considerations_2.4.pdf}
%%                 \item \url{http://christophermeiklejohn.com/distributed/systems/2013/07/12/readings-in-distributed-systems.html} 
%%             \end{itemize}
%% 
%%         }
%%         \frame{
%%             \begin{center}
%%                 \includegraphics[height=0.6\textheight]{imgs/eso-es-todo.png}
%%                 
%%             \end{center}
%% 
%%         }
%% 

\end{document}

\section{Introducci\'on}
En el contexto de la materia de Seguridad Inform\'atica, realizamos peque\~nas pruebas de concepto sobre la arquitectura \textbf{Android} con el fin de entender las posibilidades de explotaci\'on que tendr\'ia un Malware.

Este trabajo est\'a basado en algunas t\'ecnicas definidas en Xuxian Jiang,  Yajin Zhou  (auth.) Android Malware, y otras definidas por nosotros para poder sobrepasar algunas limitaciones de la tecnolog\'ia.

Los vectores en los que nos centramos para la realizaci\'on de este trabajo fueron:

\begin{itemize}
    \item Recolecci\'on 
    \item Fuga de informaci\'on.
\end{itemize}


Para esto definimos distintos módulos de aplicaci\'on que trabajaron sobre distintas piezas de informaci\'on, cada una con un prop\'osito espec\'ifico. Esta decisi\'on de dise\~no nos defini\'o:

\begin{itemize}
    \item El modelo de desarrollo del TP
    \item Un dise\~no poco acoplado en términos de funcionalidad
    \item Distintas aplicaciones que combinadas pueden extraer informaci\'on sensible.
\end{itemize}


La idea es generar distintos módulos que, en el futuro, puedan ser combinados en runtime para poder generar un malware con permisos elevados partiendo de módulos con permisos acotados.


Los requerimientos que resolvimos en el desarrollo de este TP fueron:
\begin{itemize}
    \item Obtener una copia de todas las im\'agenes que se encuentran en la memoria SD del dispositivo, y enviarlas a un sitio web controlado por el atacante.
    \item  Comportarse como un remote access tool (rat), es decir, poder ser administrado en forma remota
    \item  Obtener contactos (y toda su información)
    \item  Interceptar la interacci\'on del usuario con el teclado
\end{itemize}

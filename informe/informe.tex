\documentclass[a4paper,spanish]{article}
\usepackage[spanish]{babel}
\usepackage[utf8]{inputenc}	%para los acentos
\usepackage{caratula}
\usepackage{amsmath, amscd, amssymb, amsthm, latexsym, verbatim}
\usepackage{graphicx, graphics, caption}
\usepackage{fancyhdr}
% \usepackage{float, algorithmic}
% \usepackage[a4paper=true]{hyperref}
% \usepackage{algorithm}
\usepackage{multirow}

%probando para los margenes
\usepackage[top= 3cm, bottom= 3cm , left= 2.5cm, right= 2.5 cm]{geometry}

% configuro el paquete de algoritmos
% \floatname{algorithm}{Algoritmo}

\makeatletter
\newcounter{algorithmic}
\let\ORIG@algorithmic\algorithmic
\def\algorithmic{\stepcounter{algorithmic}\ORIG@algorithmic}
\def\theHALC@line{\thealgorithmic-\theALC@line}
\def\theHALC@rem{\thealgorithmic-\theALC@rem}
\makeatother

% encabezados
\newcommand{\norma}[1]{\left|\left|#1\right|\right|}
\parskip=1ex
\pagestyle{fancy}
\pagenumbering{arabic}
%\fancyhf{}
\renewcommand{\headrulewidth}{0.02 cm}
\renewcommand{\footrulewidth}{0 cm}
\rhead{Animal}
\lhead{Seguridad de la Información}
\def\septad{\rule{16 cm}{.2 mm}}
\usepackage{hyperref}

% defino un environment propio para las ecuaciones
%\newenvironment{ecuacion}
%	{\begin{equation} \begin{aligned}}
%	{\end{aligned} \end{equation}}
%	
%\newenvironment{ecuacion*}
%	{\begin{equation*} \begin{aligned}}
%	{\end{aligned} \end{equation*}}

% comienzo el documento
\begin{document}
	\materia{Seguridad de la Información}

\titulo{Animal}

\integrante{Mauricio Alfonso}{65/09}{mauricioalfonso88@gmail.com}
\integrante{Andres Ispani}{530/04}{andyispani@gmail.com}
\integrante{Ezequiel Gutesman}{715/02}{egutesman@gmail.com}
\integrante{Daniel Foguelman}{}{dj.foguelman@gmail.com}


\maketitle


	\tableofcontents
        \section{Introducci\'on}
En el contexto de la materia de Seguridad Inform\'atica, realizamos peque\~nas pruebas de concepto sobre la arquitectura \textbf{Android} con el fin de entender las posibilidades de explotaci\'on que tendr\'ia un Malware.

Este trabajo est\'a basado en algunas t\'ecnicas definidas en Xuxian Jiang,  Yajin Zhou  (auth.) Android Malware, y otras definidas por nosotros para poder sobrepasar algunas limitaciones de la tecnolog\'ia.

Los vectores en los que nos centramos para la realizaci\'on de este trabajo fueron:

\begin{itemize}
    \item Recolecci\'on 
    \item Fuga de informaci\'on.
\end{itemize}


Para esto definimos distintos módulos de aplicaci\'on que trabajaron sobre distintas piezas de informaci\'on, cada una con un prop\'osito espec\'ifico. Esta decisi\'on de dise\~no nos defini\'o:

\begin{itemize}
    \item El modelo de desarrollo del TP
    \item Un dise\~no poco acoplado en términos de funcionalidad
    \item Distintas aplicaciones que combinadas pueden extraer informaci\'on sensible.
\end{itemize}


La idea es generar distintos módulos que, en el futuro, puedan ser combinados en runtime para poder generar un malware con permisos elevados partiendo de módulos con permisos acotados.


Los requerimientos que resolvimos en el desarrollo de este TP fueron:
\begin{itemize}
    \item Obtener una copia de todas las im\'agenes que se encuentran en la memoria SD del dispositivo, y enviarlas a un sitio web controlado por el atacante.
    \item  Comportarse como un remote access tool (rat), es decir, poder ser administrado en forma remota
    \item  Obtener contactos (y toda su información)
    \item  Interceptar la interacci\'on del usuario con el teclado
\end{itemize}

	\section{Animal}
	\subsection{Android ADT Bundle} 
	
	\subsection{Genymotion Android Emulator}

	\subsection{Sinatra}
	\section{Animal}

	\subsection{Descripción}

	Animal es un malware de android que se encarga de robar las fotos del teléfono y subirlas a un servidor nuestro. 

	\subsection{Permisos requeridos}
		\begin{itemize}
			\item \texttt{android.permission.INTERNET} para acceder a internet y subir las fotos.
			\item \texttt{android.permission.ACCESS\_NETWORK\_STATE} para saber si hay wi-fi disponible.
		\end{itemize}

	\subsection{Implementación}

		\subsubsection{MainActivity}
			Animal crea una actividad de android \footnote{ \url{http://developer.android.com/reference/android/app/Activity.html} } en \texttt{MainActivity}, que busca en la tarjeta SD archivos que sean fotos sacadas con la cámara del teléfono. Para eso busca en 3 directorios donde suelen haber fotos (\texttt{DCIM/Camera}, \texttt{Pictures} y \texttt{Downloads}) archivos con nombres de fotos (que empiecen con \texttt{IMG} o \texttt{DSC} y tengan \texttt{jpg}, \texttt{png} o \texttt{jpeg} como extensión). Una vez que encontró todos las fotos ejecuta varios threads de la tarea asincrónica \texttt{FileUploader}, que se encarga de subir una foto cada uno. Los threads se acumulan en una cola de manera que no haya más de uno corriendo al mismo tiempo. Una vez que terminó de agregar una llamada a ejecución de \texttt{FileUploader} por cada foto, cierra la actividad haciendo una llamada al método \texttt{finish()}. De esta manera cuando el usuario abre la aplicación, ésta se cierra casi inmediatamente pero \texttt{FileUploader} sube las fotos en segundo plano. 

		\subsubsection{FileUploader}
			Cada thread de la tarea asincrónica \texttt{FileUploader} se encarga de subir la foto que reciba, haciendo un \texttt{POST} \texttt{HTTP} hacia nuestro servidor. Cada ejecución de un thread recibe un archivo como parámetro. Para evitar subir archivos usando 3G, \texttt{Animal} chequea si el celular está conectado a internet por wi-fi, y en caso contrario espera hasta que lo esté, chequeando el estado de conexión periódicamente. Una vez que el celular se encuentra conectado a una red wi-fi, el thread envía un \texttt{POST HTTP} a la dirección de nuestro servidor. Dicho \texttt{POST} contiene el nombre y contenido de la foto codificados en Base64 como parámetros de url \texttt{nombre} y \texttt{valor} respectivamente. El modo de codificación de Base64 no es el tradicional, sino uno conocido como \texttt{url-safe Base64} que no contiene ningún caracter no permitido en un url ni fines de línea.\\

			Como \texttt{FileUploader} es una \texttt{AsyncTask} \footnote{\url{http://developer.android.com/reference/android/os/AsyncTask.html}}, cada uno de los llamados a ejecución que se hacen se guardan en una cola y se ejecutan de manera serial, por lo tanto los POSTs no se envían todos juntos en paralelo, lo cuál sería malo para nuestro servidor y sobrecargaría la CPU del celular con threads si hubiera muchas fotos.
		
		\subsubsection{Servercito}
			El servidor escucha los POSTs que recibe a la ruta \texttt{/uploads}, decodifica el nombre y contenido del archivo y lo guarda en la carpeta \texttt{/fotos}.

	\subsection{Problemas}
		Intentamos en principio subir las fotos en Base64 a Pastebin \footnote{ \url{http://pastebin.com} } y poner una ``clave'' como nombre de las subidas, para poder encontrarlas con facilidad de manera anónima sin tener que exponer la dirección de nuestro servidor, pero nos encontramos con 2 limitaciones importantes de Pastebin: 
		\begin{itemize}
			\item Un l\'imite de tamaño de 0.5 MB nos limita a fotos muy chicas o a achicar las fotos existentes
			\item Un l\'imite de 10 subidas por d\'ia.
		\end{itemize}

		Por estas razones preferimos usar un servidor propio.


	\section{Animal}
Animalazo modulo malware de Android que roba los contactos de la v\'ctima y los sube a un servidor.

El proceso de instalaci\'on est\'a pensado para montarse sobre una aplicaci\'on inofensiva.
Esta invita al usuario a pelear por salvar a los animales, mostrando dos perritos corriendo. El usuario ignora que dentro de esta aplicac\'on se encuentra un mecanismo que fuga informaci\'on privada de los usuarios.

\subsection{Permisos}

El \'unico permiso que requiere es acceso de lectura a la los contactos (\texttt{android.permission.READ\_CONTACTS}). Uno de los permisos m\'as utilizados por malware.

Para la fuga de informaci\'on no precisamos m\'as permisos. Esto es gracias a la facilidad que tiene Android de poder abrir urls arbitrarias en el browser.

\subsection{Funcionamiento}
Cuando el usuario hace click en el bot\'on para ``ayudar'' el m\'odulo realiza las siguientes operaciones:
\begin{itemize}
    \item Lee la base de datos de contactos
    \item Se genera una estructura con los contactos aplanados
    \item Se transforman estos datos encodeandolos en bsae64
    \item Se crea un Intent\footnote{\url{http://developer.android.com/reference/android/content/Intent.html}}que abre un browser como si se estuviese visitando una p\'agina.
    \item Se hace un GET a un server controlado por el atacante, enviando por los par\'ametros del GET los contactos encodeados (hay limitaciones de tama\~no)
    \item El server malicioso responde con un Redirect (302) a google, como si nada hubiera pasado.
\end{itemize}

\subsection{Problemas encontrados}
A veces la aplicaci\'on no enviaba los contactos, pensamos que tenia cacheada la URL del Intent. Despu\'es nos dimos cuenta que los distintos estados de la aplicaci\'on eran relevantes: \texttt{onCreate()}, \texttt{onStart()}, \texttt{onResume()}. Afinando mejor pudimos entender cual era el problema.

\subsection{Mejoras a\'un no implementadas}
Limitaci\'on en el tama\~no de los URIs utilizados para extraer la informaci\'on. Aunque no est\'a definido este l\'imite de manera expl\'icita en el RFC2616\footnote{\url{http://www.faqs.org/rfcs/rfc2616.html}} hay limitaciones en las implementaciones.


Tambi\'en teniendo acceso a la base de datos de contactos podriamos obtener las cuentas a las q cada contacto est\'a asociado, obteniendo de esta manera todas las cuentas que se encuentran instaladas en el dispositivo sin necesidad de tener permisos de \texttt{Account}.


\subsection{Instrucciones}

Levantar el servidor seg\'un las instrucciones definidas en la \'ultima secci\'on.

Instalar la aplicaci\'on haciendo "Play" en el proyecto en \texttt{Eclipse}

Dentro del emulador, con contactos cargados, abrir la aplicaci\'on.

Hacer click en la opci\'on de \texttt{Ayudar}

Listar los contactos guardados en out\_files/



\end{document}

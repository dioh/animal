\section{Animal}

	\subsection{Descripción}

	Animal es un malware de Android que se encarga de robar las fotos del teléfono y subirlas a un servidor nuestro. 

	\subsection{Permisos requeridos}
		\begin{itemize}
			\item \texttt{android.permission.INTERNET} para acceder a internet y subir las fotos.
			\item \texttt{android.permission.ACCESS\_NETWORK\_STATE} para saber si hay wi-fi disponible.
		\end{itemize}

	\subsection{Implementación}

		\subsubsection{MainActivity}
			Animal crea una actividad de Android \footnote{ \url{http://developer.android.com/reference/android/app/Activity.html} } en \texttt{MainActivity}, que busca en la tarjeta SD archivos que sean fotos sacadas con la cámara del teléfono. Para eso busca en 3 directorios donde suelen haber fotos (\texttt{DCIM/Camera}, \texttt{Pictures} y \texttt{Downloads}) archivos con nombres de fotos (que empiecen con \texttt{IMG} o \texttt{DSC} y tengan \texttt{jpg}, \texttt{png} o \texttt{jpeg} como extensión). Una vez que encontró todos las fotos ejecuta varios threads de la tarea asincrónica \texttt{FileUploader}, que se encarga de subir una foto cada uno. Los threads se acumulan en una cola de manera que no haya más de uno corriendo al mismo tiempo. Una vez que terminó de agregar una llamada a ejecución de \texttt{FileUploader} por cada foto, cierra la actividad haciendo una llamada al método \texttt{finish()}. De esta manera cuando el usuario abre la aplicación, ésta se cierra casi inmediatamente pero \texttt{FileUploader} sube las fotos en segundo plano. 

		\subsubsection{FileUploader}
			Cada thread de la tarea asincrónica \texttt{FileUploader} se encarga de subir la foto que reciba, haciendo un \texttt{POST} \texttt{HTTP} hacia nuestro servidor. Cada ejecución de un thread recibe un archivo como parámetro. Para evitar subir archivos usando 3G, \texttt{Animal} chequea si el celular está conectado a internet por wi-fi, y en caso contrario espera hasta que lo esté, chequeando el estado de conexión periódicamente. Una vez que el celular se encuentra conectado a una red wi-fi, el thread envía un \texttt{POST HTTP} a la dirección de nuestro servidor. Dicho \texttt{POST} contiene el nombre y contenido de la foto codificados en Base64 como parámetros de url \texttt{nombre} y \texttt{valor} respectivamente. El modo de codificación de Base64 no es el tradicional, sino uno conocido como \texttt{url-safe Base64} que no contiene ningún caracter no permitido en un url ni fines de línea. La dirección del servidor se encuentra en el archivo \texttt{server.properties} de la carpeta \texttt{assets} del proyecto.\\
			
			Como \texttt{FileUploader} es una \texttt{AsyncTask} \footnote{\url{http://developer.android.com/reference/android/os/AsyncTask.html}}, cada uno de los llamados a ejecución que se hacen se guardan en una cola y se ejecutan de manera serial, por lo tanto los POSTs no se envían todos juntos en paralelo, lo cuál sería malo para nuestro servidor y sobrecargaría la CPU del celular con threads si hubiera muchas fotos.
		
		\subsubsection{Servercito}
			El servidor escucha los POSTs que recibe a la ruta \texttt{/uploads}, decodifica el nombre y contenido del archivo y lo guarda en la carpeta \texttt{/fotos}.

	\subsection{Problemas}
		Intentamos en principio subir las fotos en Base64 a Pastebin \footnote{ \url{http://pastebin.com} } y poner una ``clave'' como nombre de las subidas, para poder encontrarlas con facilidad de manera anónima sin tener que exponer la dirección de nuestro servidor, pero nos encontramos con 2 limitaciones importantes de Pastebin: 
		\begin{itemize}
			\item Un l\'imite de tamaño de 0.5 MB nos limita a fotos muy chicas o a achicar las fotos existentes
			\item Un l\'imite de 10 subidas por d\'ia.
		\end{itemize}

		Por estas razones preferimos usar un servidor propio.

